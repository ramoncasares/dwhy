% dwhy.tex (RMCG20110323)
% (c) by www.ramoncasares.com 2011
% License: cc-by-sa
\input doc
\rmcglayout


\title0 dwhy

\title1 Syntax

Dash script "dwhy" uses syntax: "dwhy name".
It tells me why the object named "name" is in my Debian system.
Object "name" can be a file, or a program, or a package.

\title1 Examples

I will explain "dwhy" showing some examples.

\title2 File

I will learn first why file "/etc/crontab" is in my system.
\begincode
$ dwhy /etc/crontab
/etc/crontab: ASCII English text
Package: pool/main/c/cron/cron_3.0pl1-116_amd64.deb [important]
 cron (M) <- base system (D)
\endcode
So "dwhy" tells me that "/etc/crontab" is a text file,
which is part of package "cron", version "3.0pl1-116",
for the architecture "amd64", and priority "important".
Finally, it tells me that 
package "cron" was manually "(M)" installed
as part of Debian "(D)" base system.

\title2 Program

Now, I will try "dwhy" with a program, for example "file".
\begincode
$ dwhy file
file -> /usr/bin/file
/usr/bin/file: ELF 64-bit LSB executable, x86-64, [...]
Package: pool/main/f/file/file_5.04-5_amd64.deb [standard]
 file (M) <- standard system (D)
\endcode
Now, "dwhy" first locates the executable file, "/usr/bin/file",
and then explains that it is part of package "file", priority "standard",
installed by Debian "(D)" as part of the standard system.

\title2 Package

If I apply "dwhy" to a package installed in my system,
for example "samba", the session goes as follows:
\begincode
$ dwhy samba
samba: file not found, try as package
Package: pool/main/s/samba/samba_3.5.6~[...]_amd64.deb [optional]
 samba (M)
\endcode
It means that I have installed package "samba" myself manually "(M)".

\title2 Directory

If I use a directory as argument, I also get some information.
\begincode
$ dwhy /etc/network
/etc/network: directory
Package: pool/main/n/netbase/netbase_4.45_all.deb [important]
 netbase (M) <- base system (D)
Package: pool/main/w/wpasupplicant/wpasupplicant_[...].deb [optional]
 wpasupplicant (A) <- network-manager-gnome network-manager 
 network-manager-gnome (A)
 network-manager (A) <- network-manager-gnome (A)
Package: pool/main/i/ifupdown/ifupdown_0.6.10_amd64.deb [important]
 ifupdown (M) <- base system (D)
Package: pool/main/v/vde2/vde2_2.2.3-3_amd64.deb [optional]
 vde2 (A)
Package: pool/main/s/samba/samba_3.5.6~[...]_amd64.deb [optional]
 samba (M)
Package: pool/main/o/openssh/openssh-server_[...]_amd64.deb [optional]
 openssh-server (M) <- ssh-server (T)
Package: pool/main/s/sysvinit/initscripts_[...]_amd64.deb [required]
 initscripts (M) <- base system (D)
Package: pool/main/a/avahi/avahi-daemon_[...]_amd64.deb [optional]
 avahi-daemon (M) <- desktop (T)
\endcode
The directory is used by several packages,
so "dwhy" has to explain why each one is installed.
Here I get all kind of explanations.
Package "wpasupplicant" was installed automatically "(A)",
either by pachage "network-manager-gnome" or package "network-manager",
so "dwhy" investigates each one.
Package "vde" was installed automatically "(A)",
though it does not depend on any other pakage;
this means that it was installed as a recommendation.
Package "openssh-server" was installed as part
of the task "(T)" "ssh-server".

\title2 Out of Debian

Of course, Debian is not responsible of every file in my system.
\begincode
$ dwhy dwhy 
/home/user/src/dash/dwhy/dwhy: a /bin/dash script text executable
dpkg: /home/user/src/dash/dwhy/dwhy not found.
/home/user/src/dash/dwhy/dwhy is not in a Debian package
\endcode

And, of course, I have not installed every Debian package.
\begincode
$ dwhy kde
kde: file not found, try as package
No packages found matching kde.
Package: pool/main/m/meta-kde/kde_66_all.deb []
No packages found matching kde.
 kde (X)
\endcode
It means that package "kde" is not "(X)" installed in my system.

Finally, of course, sometimes I fail to ask a proper question.
\begincode
$ dwhy foo
foo: file not found, and it is not a package.
\endcode


\title1 Problems

One of the problems was already presented:
"dwhy" cannot explain why packages installed as recommendations
are in the system.

Another problem is that some files installed by Debian,
as "/etc/fstab", are not part of any package,
though they are built by a package installation script,
and "dwhy" does not do a good job with these.
\begincode
$ dwhy /etc/fstab
/etc/fstab: ASCII English text
dpkg: /etc/fstab not found.
/etc/fstab is not in a Debian package
\endcode

Speed can become a problem. I wrote "dwhy" on the basis of
other programs, "dpkg-query", "apt-cache", "aptitude", etc.,
instead of going straight to the databases in
"/var/lib/dpkg", "/var/lib/apt", and "/var/lib/aptitude".


\title1 The code

Magicians advise not to reveal the tricks,
but Debian is not magic.

\file{dwhy}


\bye

